%!iTeXMac(project): pdflatex
%! iTeXMac(mark): ADDED

\documentclass{acm-final/sig-alternate-modified}
\usepackage{parallel}

\newif\ifpdf
\ifx\pdfoutput\undefined
\pdffalse % we are not running PDFLaTeX
\else
\pdfoutput=1 % we are running PDFLaTeX
\pdftrue
\fi

\ifpdf
\usepackage[pdftex]{graphicx}
\else
\usepackage{graphicx}
\fi

% HC - Always use the EPS version
%\DeclareGraphicsExtensions{.eps}

\ifpdf
\DeclareGraphicsExtensions{.pdf, .jpg}
\else
\DeclareGraphicsExtensions{.eps, .jpg}
\fi

\newcommand{\stt}[1]{\scriptsize {\tt #1}}

\begin{document}

%\conferenceinfo{ICSOC'04,} {November 15--19, 2004, New York, New York, USA.}  
%\CopyrightYear{2004}  
%\crdata{1-58113-871-7/04/0011}


\title{llava - Java in Lisp Syntax}

\numberofauthors{1}

\author{
\alignauthor Harold Carr \\
       \email{carr@llava.org}
}

\toappearbox{Submitted to the International Lisp Conference 2005}

\maketitle

%======================================================================
\begin{abstract}

{\tt llava} is Java in Lisp (lack of) syntax (rather than a Lisp or
Scheme written in Java).  {\tt llava} does not contain special syntax
or functions to call Java methods nor does it define an orthogonal set
of types (such as Scheme strings or Common Lisp arrays).  Instead,
{\tt llava} is Java expressed in typical prefix list syntax with all
data being native Java data types (e.g., instances of Java classes).
{\tt llava} adds additional types (e.g., {\tt Pair}, {\tt Procedure},
{\tt Symbol} and {\tt Syntax}) to enable one to work with lists and to
define procedures and macros.

This paper discusses the motivation for Java, presents how it differs
from similar efforts, and describes its implementation: the
Reflective-Invocation system that is the core of enabling {\tt llava}
to provide a prefix notation for Java method calls; how {\tt llava}
procedures (e.g., lambda) are integrated with Java such that method
invocations on Java objects take precedence over {\tt llava}
procedures; and extending {\tt package} and {\tt import} to work with
both Java class files and {\tt llava} procedures and files; and
finally, the {\tt llava} compilation strategy and the ``engine''-based
runtime.

The rest of this version of this paper is an extended abstract.

\end{abstract}

%======================================================================

%\category{C.2.4}{Computer-Communication Networks}{Distributed Systems}
%\category{D.2.11}{Software Engineering}{Software Architectures}[patterns - remoting]

%\terms{Design}

%\keywords{Middleware, Adaptive, RPC, RMI, Messaging, IIOP, SOAP}

%======================================================================
\section{Overview}

%======================================================================
\subsection{Motivation}

I once built Lisp systems \cite{LaSC5,LaSC3}.  Then funding caused me
to do C++ \cite{DC++}.  Then Java arrived and programming felt fun
again.  It reminded me I wanted more.  So I used Kawa \cite{kawa} for
interactively and abstractions (i.e., macros).  But I kept running
into missing Scheme library items.  So I made foreign-function calls
to Java.  However, I had to convert between Kawa/Scheme I/O, strings,
characters, etc., and Java types.  (Note: this was with an early
version of Kawa.)  This lack of interoperability made me realize I
didn't want Lisp/Scheme in Java.

That's when I got the idea for {\tt llava}.  Why not create a version
of Java that uses Lisp syntax to write Java classes?  Then extend that
system to include a few special forms (e.g., {\tt lambda}, {\tt set!},
{\tt if}, {\tt quote}, {\tt define-syntax}), pairs and symbols, a REPL
and support for incremental (re)definition.

{\tt llava} \emph{is} access to Java via Lisp syntax and features.
The philosophy of {\tt llava} is: maximum leverage of Java---only add
what is missing or cannot be done in Java.

%======================================================================
\subsection{The llava Language}

{\tt llava} is a prefix version of Java expressed in list structure.
As such, it is \emph{not} a version of Scheme with special notation to
access Java (like Jscheme \cite{jscheme}) or a version of Common Lisp
with a foreign-function interface to Java (like ArmedBear Common Lisp
(ABCL) \cite{abcl}).  Table~\ref{j-l-j-a} compares language features
between Java, {\tt llava} and these other languages.  The table should
make it clear that {\tt llava} is a more direct prefix/list notation
of Java - {\tt llava}'s primary goal.

\begin{table*}
\centering
\caption{Notation Comparisons}
\begin{tabular}{|l|l|l|l|} \hline
Java		&	{\tt llava}	&	jscheme	& abcl \\ \hline

{\bf new} & & & \\ 

{\stt import java.util.Hashtable;}
                &{\stt (import java.util.Hashtable)}
                                        &{\stt (import "java.util.Hashtable")}
                                                        & \\
{\stt Hashtable ht = new Hashtable();}
                &{\stt (set! ht (new Hashtable))}
                                        &{\stt (set! ht (Hashtable.))}
                                            &{\stt (setq ht } \\
                &                       &   &\hspace{0.5em}{\stt   (jnew (jconstructor "java.util.Hashtable")))} \\

{\bf virtual method calls} & & & \\

{\stt ht.put("three", 3);}
                &{\stt (put ht "three" 3)}
                                        &{\stt (.put ht "three" 3)}

   &{\stt (jcall (jmethod "java.util.Hashtable" "put"} \\
&&&\hspace{2em} {\stt "java.lang.Object" "java.lang.Object")} \\
&&&\hspace{1em} {\stt ht "three" 3)} \\

{\bf static method calls} & & & \\

                &{\stt (import java.lang.Byte)}
                                        &
                                                     & \\
{\stt Byte b = Byte.decode("3");}
                &{\stt (set! b (Byte.decode "3"))}
                                    &{\stt (set! b (Byte.decode "3"))}
          &{\stt (setq b (jstatic } \\
 &&&\hspace{2em}{\stt (jmethod "java.lang.Byte" "decode"} \\
 &&&\hspace{2.5em}{\stt "java.lang.String")} \\
 &&&\hspace{1em}{\stt nil "3"))} \\ \hline

\end{tabular}
\label{j-l-j-a}
\end{table*}


Jscheme uses ``dot'' notation.  This seems to have one advantage over
{\tt llava}: accessing a static field can be accessed as a variable
reference, whereas {\tt llava} chooses to represent this as a
procedure call (perhaps not as ``direct'' a represent).
Table~\ref{j-l-j} shows this (along with virtual field access).

\begin{table*}
\centering
\caption{More Notation Comparisons}
\begin{tabular}{|l|l|l|} \hline
Java		&	{\tt llava}	&	jscheme	\\ \hline

{\bf virtual fields} & & \\

{\stt import org.omg.CORBA.IntHolder;}
                &{\stt (import org.omg.CORBA.IntHolder)}
  					&{\stt (import "org.omg.CORBA.IntHolder")} \\
{\stt IntHolder ih = new IntHolder();}
		&{\stt (set! ih (new IntHolder))}
					&{\stt (set! ih (IntHolder.))} \\
{\stt ih.value = 3;}
		&{\stt (value! ih 3)}
					&{\stt (.value\$ ih 3)} \\
{\stt ih.value;}
		&{\stt (value ih)}
					&{\stt (.value\$ ih)} \\
&&\\
{\bf static fields} & & \\

{\stt import java.io.File;}
		&{\stt (import java.io.File)}
					&{\stt (import "java.io.File")} \\
{\stt File.pathSeparator;}
		&{\stt (File.pathSeparator)}
					&{\stt File.pathSeparator\$} \\ \hline

\end{tabular}
\label{j-l-j}
\end{table*}

{\tt llava} presents Java directly (e.g., {\tt true} and {\tt false}
rather than Jscheme's {\tt \#t} and {\tt \#f}).  Along these lines,
rather than invent new module and condition systems, {\tt llava}
exposes {\tt package/import} and {\tt try/catch/finally} in a natural
way:

\small
\begin{verbatim}
(import java.lang.ArithmeticException)
(import java.lang.Exception)
(import java.lang.RuntimeException)

(let ((bomb 1))
  (define (demo)
    (try
       (if (< bomb 0)
           (throw (new Exception "Give up!")))
       (list "Normal result is: " (/ 2 bomb))
     (catch (ArithmeticException e)
       (list "Arithmetic: " e))
     (catch (Exception e)
       (list "Exception: " e))
     (finally
       (set! bomb (- bomb 1))))))
\end{verbatim}
\normalsize

%======================================================================
\subsection{Implementation}

%==================================================
\subsubsection{Reflective Invocation}

{\tt llava}'s Reflective Invocation (RI) system enables access to Java
without special syntax or explicit foreign-function interfaces.  Once
a Java object is obtained its methods and fields are immediately
available:

\small
\begin{verbatim}
(toLowerCase (substring "Foo-Bar" 4))
  => "bar"
\end{verbatim}
\normalsize

{\tt llava} looks up {\tt llava} procedure definitions in internal
symbol tables.  If a procedure is \emph{not} defined then control goes
to the RI.  The RI uses Java reflection to invoke the appropriate
method by using the procedure call name as the method name, the first
argument as the class type (and message recipient) and the types of
the remaining arguments.  The full paper gives details such as method
overload resolution, caching techniques and handling overloaded
methods when {\tt null} arguments are given.

%==================================================
\subsubsection{llava procedures viz Java method calls}

A {\tt llava} procedure with the same name as a method of the first
argument (and matching argument types) will be invoked if defined:

\small
\begin{verbatim}
(define toLowerCase
  (lambda (x) (toUpperCase x)))

(toLowerCase (substring "Foo-Bar" 4))
  => "BAR"
\end{verbatim}
\normalsize

Most of the time that is \emph{not} what you want to happen.  {\tt
llava} provides an alternate form of {\tt define} that allows Java
methods to take precedence:

\small
\begin{verbatim}
(define (toLowerCase x)
  (- x))

(toLowerCase (substring "Foo-Bar" 4))
  => "bar"
(toLowerCase 3.4)
  => -3.4
\end{verbatim}
\normalsize

In other words, if the lookup finds a procedure defined with an
explicit {\tt lambda} then that procedure will always be invoked
regardless of the argument types.  If lookup finds a procedure defined
\emph{without} an explicit {\tt lambda} then {\tt llava} will first
attempt an RI invocation.  If RI finds a method, it will invoke the
method return the results.  If RI does not find a method then the {\tt
llava} procedure will be invoked.  If lookup does not find either type
of {\tt llava} procedure then it will attempt an RI invocation.  If RI
does not find a method then {\tt llava} will throw a {\tt
LlavaException} indicating an {\tt undefined top-level variable}.

%==================================================
\subsubsection{package/import system}

Like Java, {\tt llava}'s {\tt import} enables ``short''-form access to
constructors and static methods and fields.  Unlike Java, {\tt llava
import} is required to enable access to virtual fields:

\small
\begin{verbatim}
(import java.lang.Long)
(set! l (new Long 34))
(getName (getClass (Long.decode "23")))
 => "java.lang.Long"
(Long.MAX_VALUE)
 => 9223372036854775807

(import java.lang.System)
(System.out)
 => java.io.PrintStream@1434234
\end{verbatim}
\normalsize

When a Java class is imported, {\tt llava} creates a {\tt llava}
package for that class and defines procedures in that package to
access fields.  It also adds the imported package to the list of
packages imported by the package doing the importing.  Then, during
variable lookup, any names with a single dot (``.'') in them will be
expanded to the full package name.

% What about foo.bar.inner to get inner classes?

{\tt llava} code may define packages:

\small
\begin{verbatim}
(package org.example)

(define (whatZone x)
  (let* ((tz (getTimeZone x))
         (n  (getDisplayName tz)))
    n))
\end{verbatim}
\normalsize

%(whatZone (DateFormat.getInstance))

The full paper shows how {\tt llava}'s {\tt package} and {\tt
import} are implemented and how they interact with Java {\tt
classpath}, the file system, and the REPL.


%==================================================
\subsubsection{Compiler and Runtime system}

{\tt llava} is compiled to a few special forms: application, {\tt
begin}, {\tt if}, {\tt lambda}, literal, ref, {\tt set!}.  Those forms
are written as explicit Java classes, each with a {\tt run} method
that takes the current lexical environment and an evaluation engine
that also has a {\tt run method}.  The forms and the engine cooperate
such that the same call to {\tt engine.run} is executed in a Java {\tt
while} loop when evaluating tail calls.  This makes the {\tt llava}
interface to Java properly tail-recursive.  More details on the
engine-based runtime are given in the full paper.  The engine
evaluator is similar to one used in older versions of Jscheme.

%==================================================
\subsection{Outline}

The main point of {\tt llava} is that it is Java in Lisp syntax rather
than another language written in Java.  It extends Java by providing
macros, procedures, lists, and symbols.  Its core feature is
Reflective Invocation enabling method calls on Java objects without
special syntax.  The remainder of this paper gives more details on the
design of the language, implemenation techniques and performance.

%======================================================================
\section{The {\tt llava} Language}

\begin{table*}
\centering
\caption{Notation Comparisons}
\begin{tabular}{|l|} \hline


\begin{Parallel}[v]{0.4\textwidth}{0.51\textwidth}
  \tolerance=1000
  \ParallelLText{Itemque ipsa adiutoria distinguenda sunt. Aliud est
    adiutorium sine quo aliquid non fit, et aliud est adiutorium quo
    aliquid fit.  \ldots}
  \ParallelRText{Und ebenso mu\ss\ man die Gnadenhilfen selbst
    unterscheiden. Etwas anderes ist eine Hilfe, ohne die etwas nicht
    geschieht, und etwas anderes eine Hilfe, durch die etwas
    geschieht.\ldots}
  \ParallelPar
  \ParallelLText{Primo itaque homini, qui in eo bono quo factus fuerat
    rectus acceperat posse non peccare, posse non mori, posse ipsum bonum
    non deserere, datum est adiutorium perseverantiae, non quo fieret ut
    perseveraret, sed sine quo per liberum arbitrium perseverare non
    posset. Nunc vero sanctis in regnum Dei per gratiam Dei praedestinatis
    non tale adiutorium perseverantiae datur, sed tale ut eis perseverantia
    ipsa donetur; non solum ut sine isto dono perseverantes esse non
    possint, verum etiam ut per hoc donum non nisi perseverantes sunt.}
  \ParallelRText{Dem ersten Menschen, der in dem Gute, worin er gerecht
    erschaffen war, die F\"ahigkeit empfangen hatte, nicht zu s\"undigen,
    nicht zu sterben und vom Guten selbst nicht ab\-zufallen, ist demnach
    die Gnade der Beharrlichkeit verliehen worden, nicht jene, wo\-durch
    seine Beharrlichkeit bewirkt worden w\"are, sondern jene, ohne die er
    nicht imstande gewesen w\"are, mit seinem freien Willen auszuharren.
    Jetzt aber wird den Heiligen, die durch die Gnade Gottes f\"ur das Reich
    Gottes vorher\-bestimmt sind, nicht eine solche Gnade der Beharrlichkeit
    gegeben, sondern eine derartige, da\ss\ ihnen die Beharrlichkeit selbst
    geschenkt wird; daher sind sie ohne dieses Gnadengeschenk nicht nur
    unf\"ahig zur Beharrlichkeit, sondern sind auch durch dieses Geschenk
    Nurbeharrende.}
\end{Parallel}
\end{tabular}
\label{point}
\end{table*}


\begin{table*}
\centering
\caption{Notation Comparisons}
\begin{tabular}{|l|l|} \hline
Java                                              & {\tt llava} \\ \hline

{\stt package org.openhc.llavademo.cp;}           & {\stt (package org.openhc.llavademo.cp)} \\

{\stt import org.openhc.llavademo.pb.PointBase;}  & {\stt (import org.openhc.llavademo.pb.PointBase)} \\

{\stt public abstract class ColoredPointBase}     & {\stt (abstract class ColoredPointBase extends PointBase} \\
{\stt    extends PointBase}                       & {\stt   (protected String color))} \\
{\stt \{}                                         & \\
{\stt    protected String color;}                 & \\
{\stt \}}                                         & \\
\end{tabular}
\label{point}
\end{table*}

\begin{thebibliography}{1}

\bibitem{LaSC5}
R.~Kessler, H.~Carr, L.~Stoller, and M.~Swanson.
\newblock Implementing Concurrent Scheme for the Mayfly distributed parallel processing system.
\newblock {\em Lisp and Symbolic Computation}, Vol 5, Issue 1-2 (May 1992) pp 73-93.
%http://portal.acm.org/citation.cfm?id=161270.161285&dl=GUIDE&dl=GUIDE&CFID=33708534&CFTOKEN=68233485

\bibitem{LaSC3}
P.~Pourheidari, R.~Kessler, and H.~Carr.
\newblock Moped (a portable debugger)
\newblock {\em Lisp and Symbolic Computation}, Vol 3, Issue 1 (January 1990) pp 39-65.
%http://portal.acm.org/citation.cfm?id=83661.83679&coll=GUIDE&dl=GUIDE&CFID=33708534&CFTOKEN=68233485

\bibitem{DC++}
H.~Carr, R.~Kessler, and M.~Swanson.
\newblock Distributed C++
\newblock {\em ACM SIGPLAN Notices}, Vol 28, Issue 1 (January 1993)
%http://portal.acm.org/citation.cfm?id=156702&jmp=indexterms&dl=GUIDE&dl=GUIDE

\bibitem{kawa}
Per Bothner
\newblock kawa
\newblock http://www.gnu.org/software/kawa/

\bibitem{jscheme}
K.~Andersen, T.~Hickey, and P.~Norvig
\newblock jscheme
\newblock http://jscheme.sourceforge.net/

\bibitem{abcl}
\newblock ArmedBear Common Lisp
\newblock http://armedbear-j.sourceforge.net/


\end{thebibliography}

\end{document}

% End of file.

